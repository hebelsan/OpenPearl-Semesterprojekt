\chapter{Macros}
\begin{discuss}
dieses ganze kapitel steht noch zur Diskussion!
\end{discuss}

\OpenPEARL{} requires the specification of the global interface 
of other modules in each module which wants to use other modules.
In some cases conditional compilation is needed. 


\section{Import Statement}

The import of a different source file is triggered with:
\#INCLUDE FileNameWithPathAndExtension

The content of the specified source file is inserted at this 
place. Error messages from compiler and runtime system will  be given 
relative to the included file.

Wirklich??? setLocation(,) muesste entsprechend umgesetzt werden.

The base location for relative path names is the directory containing 
the file, doing the INCLUDE  
If an included file contains further \#INCLUDE statements with relative
path names, they included relative to the location of the including file.
(geht sicher noch besser)

There is a limit of 7 levels of nested INCLUDE-statements.

\section{Conditional Compilation}

\subsection{Definition of Macros Constants}
\#DEFINE ConstantIdentifier {\bf = } ConstantIntegerExpression ;

ConstantIntegerExpression :== \\
\x [ ConstantMonadicOperator ] ConstantOperand  [ 
 ConstantDyadicOperator ConstantIntegerExpression  ] $^{...}$\\

ConstantMonadicOperator:== \\
\x {\bf +} $\mid$ {\bf -}

ConstantDyadicOperator:== \\
\x {\bf +} $\mid$ {\bf -} $\mid$ {\bf *} $\mid$ {\bf // }

ConstantOperand :== \\
\x Integer $\mid$ Identifier$\S $Constant $\mid$ 
{\bf (} ConstantIntegerExpression {\bf )}

Example:\\
\#DEFINE lines = 10;			\\
\#DEFINE lineLength = 80;		\\
\#DEFINE size = 2*(lines*lineLength);	\\

Example usage:

MODULE(condCompile);\\
\\
SYSTEM:\\
\ \\
PROBLEM;\\
\#INCLUDE config.inc    ! defines target\\
\ \\
\#IFUDEF target;\\
\x \#ERROR 'target is not set';\\
\#ENDIF;\\
\\
\#IF target == 2;\\
\x  \#INCLUDE 'module2.inc' ;\\
\#ELSE IF target == 3 ;\\
\x   \#INCLUDE 'module3.inc' ;\\
\#ELSE ;\\
\x   \#ERROR 'unknown target';\\
\#ENDIF;\\

Zulaessige Vergleiche: $<, <= >, >=, ==, /= $

Macros: DEFINE, INCLUDE, IF, IFUDEF, IFDEF, (?? auch UNDEF?? )

'\#' und Makroname muessen zusammen stehen. Vor einem Makro ist nur Whitespace erlaubt. Makroanweisungen enden mit ';'.
Nach dem ';' sinf normal PEARL-Anweisungen zulaessig.
