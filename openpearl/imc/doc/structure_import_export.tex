\chapter{Structure of Module Import and Export Definition}

The compiler creates an xml-file per PEARL-module containing the
\begin{itemize}
\item system part information
\item problem part specifications and declarations
\end{itemize}

\section{XML-Document Structure}
The compiler has no knowledges about the nature of a system name.
Thus the system part is translated as it is --- only syntactical errors
are recognized. 
The document has the root tag \verb|<module>|, which hat the attribute
\verb|file| with the value source file name.  
All system elements are in the xml-subtree \verb|<system>|.
All problem part elements with attribute GLOBAL are located
in the xml-subtree \verb|problem|.

\begin{XMLCode}
<?xml version="1.0" encoding="UTF-8" ?>
<module file="demo.prl">
<system>
   <username .... >
   ....
   </username>
   <configuration .... >
   ....
   </configuration>
</system>
<problem>
  <spc .../>
  <dcl .../>
  ...
</problem>
</module>
\end{XMLCode}

\section{System Part Elements}
\subsection{User Names \texttt{<username>}}
If a system element defines a user name the \verb|<username>| tag is 
used. Each user name is accompanied by an system name with the tag 
\verb|<sysname>|, which may have parameters.
Maybe there is a list of associations, thus there may by a tree
of \verb|<association>| tags.
The user name tag contains the attribute \verb|line| with the value of the 
source code line number.

\subsection{System Name \texttt{<sysname>}}
\label{sec_system_names}
The system name tag contains the attribute \verb|name| containing the 
name of the system element.

\subsection{Parameters \texttt{<parameters>}}
The parameters are located in the \verb|<parameters>|-subtree.
The imc tool supports constants of type FIXED, CHAR and BIT.
The compiler detects  the type of the parameter
from the literal and passes the literal as content to the corresponding
\verb|<FIXED>|-, \verb|<CHAR>|- or \verb|<BIT>|-tag.

\subsection{Associations \texttt{<associations>}}
An association may be ether a system name (with parameters) or a user name.
The  \verb|<association>|-tag contains the attribute \verb|name| with the 
given name as value. If parameters are specified for the association,
they are passed as \verb|<parameters>|-subtree in the same way as decribed 
in section \label{sec_system_names}.

There is no check in the compilation phase necessary,
whether a name is a system name
or a user supplied name.


\subsection{Configuration Item}
Configuration elements in the system part are located in the
\verb|<configuration>|-element. 
The only difference to the \verb|<username>|-tag is the absence of the 
\verb|name|-attribute. Parameters and associations apply identical.

\subsection{Example for System Part Elements and their Specifications}
\begin{PEARLCode}
MODULE(demo);

SYSTEM;
lm75: LM75('48'B4) --- I2CBus('/dev/i2c-0', 100000);

lm75a : LM75('49'B4) --- i2cbus1;
i2cbus1: I2CBus('/dev/i2c-1', 100000);

sig1: FixedRangeSignal;
int1: UnixSignal(15);

disc: Disc('/tmp/folder1', 10);

stdOut: StdOut;

Log('EW') --- StdOut;

PROBLEM,
SPC sig1 SIGNAL GLOBAL;
SPC int1 INTERRUPT GLOBAL;
SPC disc DATION OUT DIRECT SYSTEM ALL GLOBAL;
SPC stdOut DATION OUT SYSTEM ALPHIC GLOBAL;
SPC lm75 DATION IN SYSTEM FIXED(15) GLOBAL;

MODEND;
\end{PEARLCode}

\begin{XMLCode}
<?xml version="1.0" encoding="UTF-8" ?>
<module file="demo.prl">
<system>
   <username name="lm75" line="4">
      <sysname name="LM75">
      <parameters>
         <BIT>'48'B4</BIT>
      </parameters>
   </sysname>
   <association name="I2CBus">
      <parameters>
         <CHAR>'/dev/i2c-0'</CHAR>
         <FIXED>100000</FIXED>
      </parameters>
   </association>
</username>

<username name="lm75a" line="6">
   <sysname name="LM75">
      <parameters>
         <BIT>'49'B4</BIT>
      </parameters>
   </sysname>
   <association name="i2cbus1">
   </association>
</username>

<username name="i2cbus1" line="7">
   <sysname name="I2CBus">
      <parameters>
         <CHAR>'/dev/i2c-1'</CHAR>
         <FIXED>100000</FIXED>
      </parameters>
   </sysname>
</username>

<username name="sig1" line="9">
   <sysname name="FixedRangeSignal"/>
</username>

<username name="int1" line="10">
   <sysname name="UnixSignal">
      <parameters>
         <FIXED>15</FIXED>
      </parameters>
   </sysname>
</username>

<username name="disc" line="12">
   <sysname name="Disc">
      <parameters>
         <CHAR>'/tmp/folder1'</CHAR>
         <FIXED>10</FIXED>
      </parameters>
   </sysname>
</username>

<username name="stdOut" line="14">
   <sysname name="StdOut">
   </sysname>
</username> 

<configuration line="16">
   <sysname name="Log">
      <parameters>
         <CHAR>'EW'</CHAR>
      </parameters>
   </sysname>
   <association name="StdOut">
   </association>
</configuration>
</system>
<problem>
<spc type="signal" name="sig1" line="19"/>
<spc type="interrupt" name="int1" line="20" />
<spc type="dation" name="disc" line="21">
   <attributes> OUT,SYSTEM, DIRECT </attributes>
   <data>ALL</data>
</spc>
<spc type="dation" name="stdOut" line="22">
      <attributes> OUT, SYSTEM </attributes>
      <data>ALPHIC</data>
</spc>
<spc type="dation" name="lm75" line="23">
      <attributes> IN, SYSTEM </attributes>
      <data>A15</data>
</spc>
</problem>
</module>
\end{XMLCode}
\section{Problem Part Elements}
\subsection{Specification of System Part Elements}
The PEARL source code defined the type of a system name to be ether a DATION, 
INTERRUPT or DATION. According detected type, the compiler 
adds a xml-tag \verb|<spc>|-tag with attributes \verb|type| anf \verb|line|
containing the value \verb|"dation"|, \verb|"interrupt"| or \verb|"signal"|,
 respectively.
While interrupts and signals have no parameters, dations need more
specifications. The attributes are stores as comma separated list in the
\verb|<attributes>|-subelement, the transmission data is given as value
of the <data>-tag. For details aboutencoding of the type information
see chapter \ref{encoding}. 

\subsection{Declaration of Problem Part Elements}

Additional type information for the \verb|type| attribute are introduced.
The additional type values are identical to the  encoding rules of
structures and structure elements  as described in \cite{runtime}.

This treats all elements except dations. They are described similar to the
system part.


\subsection{Specification of Problem Part Elements}
The specification of global  elements in the problem part are described 
in the same as their declaration.

\subsection{Example for Specification and Declarations in Problem Part}

\begin{PEARLCode}
PROBLEM;
   DCL x FIXED(7) GLOBAL;
   SPC y FIXED(17) GLOBAL;
   DCL d12 DATION OUT STRUCT [a FIXED(7), b FLOAT(53)]
     DIM(10,20)  GLOBAL CREATED(aSystemDation);
   DCL s1 SEMA PRESET(10) GLOBAL;
\end{PEARLCode}

\begin{XMLCode}
...
<problem>
  <dcl type="A7" name="x" line=".."/>
  <spc type="A15" name="y" line=".."/>
  <dcl type="dation" name="d12" line="10">
    <attributes>OUT DIM(10,20)</attributes>
    <data>S5A7B57</data>
  </dcl>
  <dcl type="I" name="s1" line="..."/>
</problem>
\end{XMLCode}


\section{Encoding of Transfer Data Types and STRUCTs}
\label{encoding}
The encoding of transfer data is like described in the runtime documentation
\cite{runtime}. The primitive data types and their length are encoded 
by a letter and an integer constant. E.g A7 is a FIXED(7). The encoding rules
for struct and arrays hold. Special values are ALL and ALPHIC.

\begin{description}
\item[ALL] fits to all data types
\item[ALPHIC] in problem part fits to ALPHIC, CHAR and ALL in system devices
\end{description}

This mapping is also used for problem part global elements, except for dations,
which need more information to be checked.
