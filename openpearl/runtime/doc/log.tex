\chapter{Log}
The logging facility itself is plattform specific. The interface is 
defined plattform independently in {\em Log.h}.

The following log levels are defined:
\begin{description}
\item[INFO] general information about the program execution
\item[DEBUG] messages used for debugging the runtime system. Note that
   many messages may affect the application execution.
\item[WARN] messages aabout situations, where a backup solution is used.
E.g. start of the application without root priviledes in Linux will 
forbid the usage of the priority scheduler. The normal scheduler will be 
used instead.
\item[ERROR] are diagnostic information is case of signal raising.
\end{description}

The log interface defines the methods
\verb|Log::info()|, \verb|Log::debug()|, Log::warn()| and \verb|Log::error()|.
Each method takes at least on string argument like printf.
Plain text is passed to the corresponding output.
The following format elements are in use: \verb|%d|, \verb|%u|, \verb|%s|,
\verb|%f| and \verb|%c|. No additional parameters are allowed with these 
formatting options.

The formatting is written plattform independently.
The output is still plattform specific.
