\section{Basic Priniciples of Operation}

\subsection{System Overview}
FreeRTOS is used as scheduler. Processor specific parts are
realized with the help of LPCOpen. LTO ist used.

All system specific parts are located in \texttt{./lpc1768}. 
These componts are:
Makefile, linker script, peripheral dependent stuff like real time clock,..

All objects are packed in the PEARL-library 
\verb|libOpenPEARLlpc1768.a|. The \texttt{weak} symbols are
packed into \texttt{OpenPEARLlpc1768\_board.o} in order to overcome the 
linkage problem with LTO.

\subsection{System Initialization}
The system initialization  is controlled by the following files:
\begin{description}
\item[lpc1768/OpenPEARLlpc1768.ld] is the linker script. 
   This files is derived from
   the samples from the Launchpad cross-tool-chain file gcc.ld. 
   Only the ram and rom sizes were modified to fit the LPC1768 
\item[lpc1768/startup\_lpc1768.S] contains the interrupt vector table and the 
   reset handling routine. The reset handler 
   \begin{enumerate}
   \item copies the preset values from flash memory to RAM
   \item clears the BSS section 
   \item  calls the SystemInit() function
   \item  branches to the initialization of glibc (\_start); glibc
      initializes inself and the static objects and branches into \verb|main()|
   \end{enumerate}
\item[lpc1768/lpc17\_SystemInit.c] contains the actions, which must be executed
   before the clib is started. Up to now only the systems core clock is 
   set.
\item [lpc1768/lpc17\_uart\_retarget.c] contains some functions, which redirects console 
   input and output to the UART0.  This allows the use of printf(), 
   scanf(), cin and cout in the application. The UART initialization is done 
   upon first usage of any i/o-operation of this module.
\item[FreeRTOS/PEARL/main.cc] contains the application start code.
 The major steps are:
  \begin{enumerate}
  \item send a list of defined tasks to the Log::debug-interface
  \item activate all \verb|MAIN| tasks according their priority
  \item initialize the real time clock (RTC)
  \item initialize the software time from the RTC
  \item start the FreeRTOS scheduler 
  \end{enumerate}
  The termination is done in the class \verb|TaskMonitor|. 
\end{description}

\subsection{Interrupt Levels}
The Cortex-M systems provide different interrupt levels for peripheral devices.
The FreeRTOS system requires that interrupt service routines, which call
FreeRTOS function have the interrupt level 
\texttt{configMAX\_SYSCALL\_INTERRUPT\_PRIORITY} or worse. 
This must be regarded in each driver of a peripheral device.



