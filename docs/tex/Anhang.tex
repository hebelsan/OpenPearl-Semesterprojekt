\documentclass[12pt,a4paper]{article}
\usepackage[utf8]{inputenc}
\usepackage[german]{babel}
\usepackage[T1]{fontenc}
\usepackage{amsmath}
\usepackage{amsfonts}
\usepackage{amssymb}
\usepackage{graphicx}
\usepackage{listings}
\usepackage{color}
\usepackage{hyperref}


\definecolor{dkgreen}{rgb}{0,0.6,0}
\definecolor{gray}{rgb}{0.5,0.5,0.5}
\definecolor{mauve}{rgb}{0.58,0,0.82}

\lstset{frame=tb,
  language=Java,
  aboveskip=3mm,
  belowskip=3mm,
  showstringspaces=false,
  columns=flexible,
  basicstyle={\small\ttfamily},
  numbers=none,
  numberstyle=\tiny\color{gray},
  keywordstyle=\color{blue},
  commentstyle=\color{dkgreen},
  stringstyle=\color{black},
  breaklines=true,
  breakatwhitespace=true,
  tabsize=3
}
\author{Alexander Hebel}
\begin{document}
\section{Anhang}
\subsection{Quellcode}\label{Quellecode}
Der komplette Code der Anwendung befindet sich im Githubprojekt.\\
Link: \url{https://github.com/hebelsan/OpenPearl-Semesterprojekt/tree/master/Project}\\
In diesem Ordner befinden sich zwei Programme.\\
Ein Programm mit neu entwickelten Ausgabe und ein Programm ohne bearbeitet Ausgabe.\\
\subsection{Abgeändertes OpenPearl-Projekt}\label{abgeänderte OpenPearl-Projekt}
Das von uns abgeänderte OpenPearl-Projekt befindet sich unter folgendem Link.\\
\url{https://github.com/hebelsan/OpenPearl-Semesterprojekt/tree/master/openpearl}\\
\subsection{Aktivitätsdiagramm}\label{Aktivitätsdiagramm}
Das Aktivitätsdiagramm befindet sich ebenfalls im Repository unter \url{https://github.com/hebelsan/OpenPearl-Semesterprojekt/tree/master/docs/Activity_Diagram}.\\
\newpage
\subsection{Änderungen der Makefile im Ordner runtime}\label{Änderungen der Makefile im Ordner runtime}
\begin{figure}[h]
\begin{center}
\includegraphics[width=9cm]{grafiken/Makefile_runtime1.png}
\end{center}
\end{figure}
\clearpage
\begin{figure}[h]
\begin{center}
\includegraphics[width=11cm]{grafiken/Makefile_runtime2.png}
\end{center}
\end{figure}

\begin{figure}[h]
\begin{center}
\includegraphics[width=15cm]{grafiken/Makefile_runtime3.png}
\end{center}
\end{figure}
\clearpage
\begin{figure}[h]
\begin{center}
\includegraphics[width=16cm]{grafiken/Makefile_runtime4.png}
\end{center}
\end{figure}

\begin{figure}[h]
\begin{center}
\includegraphics[width=16cm]{grafiken/Makefile_runtime5.png}
\end{center}
\end{figure}
\clearpage
\subsection{Änderungen der Makefile im Ordner stm32f7}\label{Änderungen der Makefile im Ordner stm32f7}
\begin{figure}[h]
\begin{center}
\includegraphics[width=13cm]{grafiken/Makefile_stm32f7_1.png}
\end{center}
\end{figure}

\begin{figure}[h]
\begin{center}
\includegraphics[width=13cm]{grafiken/Makefile_stm32f7_2.png}
\end{center}
\end{figure}
\newpage

\begin{figure}[h]
\begin{center}
\includegraphics[width=35cm]{grafiken/Makefile_stm32f7_3.png}
\end{center}
\end{figure}
\newpage



\end{document}
