\section{Das Scheitern der Portierung auf den STM32F746 DISCO}
Einer der Gründe war wie schon bemerkt das Scheitern bei dem Umgang mit dem Debugger. Durch den fehlenden Debugger war es nicht möglich, den geschriebenen Code zu analysieren. Im Klartext bedeutet das, dass nicht festgestellt werden konnte, ob FreeRTOS richtig lauffähig war/richtig konfiguriert war. Da FreeRTOS das Grundgerüst für OpenPEARL darstellt, ist die Sicherstellung der Funktionstüchtigkeit aber zwingend notwendig. Um die Überprüfung des Shedulers durchführen zu können, muss man in der Lage sein, den Prozessor zu stoppen, Zwischenergebnisse, Registerinhalte anzeigen zu können, zu sehen, in welcher Zeile des Quellcodes man sich gerade befindet. Und ebendies war wegen dem Fehlen des Debuggers nicht möglich. 
Des Weiteren wurde festgestellt, dass die Toolchain, die im Projekt Verwendung fand vom Funktionsumfang her nicht geeignet war. Der Kompiler konnte Dinge nicht, die zum Übersetzen erforderlich waren.
Zu allem Übel kam hinzu, dass dem Team die Zeit knapp wurde. Bis Anfang Dezember konnten keine großartigen Ergebnisse geliefert werden, was auch dazu führte, dass die Lust am Projekt und an der Beteiligung drohte abzunehmen. 
Aus diesem Grund wurde an einer Sitzung im Dezember besprochen, wie weiter fortzufahren ist. Das Team wurde sich einig über ein Alternativprogramm.
Die Alternative sieht eine zweigleisige Entwicklung vor. Ein Teil des Teams befasst sich mit der Entwicklung der ursprünglich angedachten Anwendung. Diese Anwendung ist wichtig, da sie bei der Präsentation gegen Ende des Semesters das Projekt von einem rein theoretischen Ansatz her mit der Praxis verbindet. Die Anwendung soll auch darstellen, dass OpenPEARL wirklich funktioniert und praxistauglich ist. Die zweite Schiene der Entwicklung befasst sich mit der Entwicklung eines I2C-Treibers für das Microcontrollerboard LPC1768, das teilweise vom OpenPEARL-Projekt bereits Unterstützung fand. Beispielsweise waren schon Treiber für die Benutzung des Displays mit unterschiedlichen Schriftarten vorhanden.
