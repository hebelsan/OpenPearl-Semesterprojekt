\documentclass[12pt,a4paper]{article}
\usepackage[utf8]{inputenc}
\usepackage[german]{babel}
\usepackage[T1]{fontenc}
\usepackage{amsmath}
\usepackage{amsfonts}
\usepackage{amssymb}
\usepackage{hyperref}
\usepackage{listings}
\author{Jonathan Weißenberger}
\begin{document}
\section{Einrichtung des Debuggers}
In diesem Kapitel wird im Stil eines Tutorials vorgestellt, wie wir vorgegangen sind, um den Debugger zum Laufen zu bekommen. Als Betriebssystem kam Debian zum Einsatz.
\subsection{Udev-Rules Anpassung}
\subsection{Download Eclipse}
Um den Debugger zum Laufen zu bekommen, benötigt man zuallererst eine funktionierende Eclipse IDE. Für unser Projekt haben wir uns für 'Eclipse IDE for C/C++ Developers' entschieden erhältlich auf \url{http://www.eclipse.org/downloads/}.
Nach dem Download muss das Paket entpackt werden. In der Kommandozeile funktioniert dies mittel des Befehls 
\begin{lstlisting} 
$ tar -xzf <eclipse-download-file>
\end{lstlisting}
oder ganz einfach in der grafischen Benutzeroberfläche des Betriebssystems.
\subsection{Toolchain-Installation}
Um später debuggen zu können, wird ein Kompiler für die Zielplattform benötigt und der spezifische GDB. Diese Tools findet man in Toolchains. Ein Toolchain ist eine komplette Suite, mit Werkzeugen, um für fremde Hardwarearchitekturen Software bauen zu können.
Wir verwenden die GNU-Toolchain. Zunächst laden wir die komplette Toolchain herunter.
\begin{lstlisting}
$ cd /opt
$ sudo wget https://launchpad.net/gcc-arm-embedded/5.0/5-2016-q3-update/+download/gcc-arm-none-eabi-5_4-2016q3-20160926-linux.tar.bz2
\end{lstlisting}
Anschließend wird die Datei entpackt.
\begin{lstlisting}
$ sudo tar -xvjf gcc-arm-none-eabi-4_8-2014q2-20131204-linux.tar.bz2
\end{lstlisting}
Falls die Toolchain-Tools im späteren Entwicklungsprozess im Terminal von überall her gefunden werden sollen, empfielt sich die Anpassung der Datei ~/.bashrc. Dort wird dann einfach die PATH-Variable erweitert.
\begin{lstlisting}
$ nano ~/.bashrc
# In der Datei zu ergaenzen:
# export PATH=$PATH:/opt/<Name der entpackten tar>/bin
\end{lstlisting}
\subsection{Installieren der Eclipse Toolchain Tools}
\subsection{Installieren von OpenOCD}
\subsection{Debuggen des ersten Projekts}
\end{document}