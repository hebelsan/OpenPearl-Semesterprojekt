\chapter{Warum die Applikation auf dem LPC1768 nicht lief}
\sectionauthor{Autor: Jonathan Weißenberger}
Gegen Ende des Semesters wurde glücklicherweise der I2C-Treiber für den LPC fertiggestellt. Das Bauen unserer Applikation, die auf dem Raspberry Pi bereits lief glückte nach Bekanntmachung des Treibers im IMC auch. Jedoch wurden -wie sich später herausstellte- beim Ausführen der Applikation nicht alle Error-Meldungen und Log-Nachrichten auf der Konsole ausgegeben, sodass ein schnelles Lösen des Problems nicht möglich war. Das aleinige Ausführen der Applikation auf dem LPC1768 funktionierte nicht. Der betreuende Professor meinte, dass die Fehler tief in OpenPEARL verwurzelt seien und diese nicht auf kurze Zeit zu finden wären. Da diese Feststellung in der vorletzten Projektwoche erfolgt, war somit das Ende der Entwicklung des Treibers besiegelt.
